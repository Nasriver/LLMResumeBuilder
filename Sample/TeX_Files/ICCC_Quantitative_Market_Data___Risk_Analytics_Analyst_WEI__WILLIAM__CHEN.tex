\documentclass[10pt,letterpaper]{article}
\usepackage[utf8]{inputenc}
\usepackage[margin=0.5in]{geometry}

% Disable global paragraph indentation
\setlength{\parindent}{0pt}

% Custom section formatting
\newcommand{\cvsection}[1]{
    \vspace{1.5ex}
    {\noindent\textbf{\large \uppercase{#1}}}
    \vspace{0.3ex}
    \hrule
    \vspace{0.8ex}
}

% Custom list spacing for perfect indentation and ZERO top spacing
\renewenvironment{itemize}{
    \begin{list}{\textbullet}{
        \setlength{\leftmargin}{0.15in}
        \setlength{\itemsep}{1pt}
        \setlength{\parskip}{0pt}
        \setlength{\parsep}{0pt}
        \setlength{\topsep}{0pt}
        \setlength{\partopsep}{0pt}
    }
}{
    \end{list}
}

% Experience entry: \cventry{Company}{Location}{Role}{Dates}
% Uses tabular* to guarantee dates are flush to the right margin.
\newcommand{\cventry}[4]{%
  \noindent\begin{tabular*}{\textwidth}{@{}l@{\extracolsep{\fill}}r@{}}
    \textbf{#1} & #2 \\[0pt]
    \textit{#3} & #4 \\[0pt]
  \end{tabular*}%
}

\pagestyle{empty}

\begin{document}

\begin{center}
    {\LARGE \textbf{WEI (WILLIAM) CHEN}} \\
    999-999-9999 | wchen0924\_NOT\_REAL@gmail.com | https://www.linkedin.com/in/weiwchen\_NOT\_REAL/
\end{center}

\cvsection{Education}
\textbf{UCBB Madison School of Management} \hfill Bar Bara, CA \\
Master of Financial Engineering \hfill GPA: 3.38/4 \hfill December 2025 \\
\textit{Relevant Courses: Econometrics, Fixed-Income Markets, and Machine Learning}

\vspace{0.8ex}

\textbf{The English University of Hong Kong} \hfill Hong Kong \\
B.A., Business Administration, Minor in Statistics \hfill December 2022 \\
\textit{Relevant Courses: Time Series, Non-Parametric Statistics, and Deep Learning}

\vspace{1ex}

\cvsection{Skills \& Professional Certifications}
\textbf{Technical Skills:} Python (NumPy, Pandas, Scikit-learn, Polars and Numba), SQL, and Git \\
\textbf{Certifications:} SOA Exam QFIQF \\
\textbf{Other Skills:} Monte Carlo Simulation, Prompt Engineering, and Data Cleaning

\cvsection{Professional Experience}
\cventry{Proton Capital Management}{United States}{Quantitative Research Intern}{June 2025 -- September 2025}
\begin{itemize}
\item Architected a systematic fixed-income screening workflow in Python using yield-spread time series across 500+ instruments, improving data integrity for relative-value identification and downstream risk analysis.
\item Built a scenario-based rate stress-testing engine to measure DV01 and convexity under parallel shifts, twists, and butterfly moves, supporting sensitivity analysis aligned with enterprise market risk reporting.
\item Streamlined daily risk data production by programmatically aggregating Greeks and generating reports, reducing manual runtime by 60\% and eliminating reconciliation errors via consistent validation logic.
\end{itemize}
\vspace{1ex}

\cventry{General Life Insurance}{Hong Kong}{Actuarial Analyst}{January 2023 -- September 2024}
\begin{itemize}
\item Conducted deep statistical analysis on 200K+ policy records to quantify lapse and mortality behavior, strengthening assumption governance and improving reliability of IFRS 17 reserving inputs.
\item Automated model validation pipelines in Python by replacing manual Excel controls with repeatable checks, shortening the quarterly reporting cycle by 40\% while improving traceability and documentation quality.
\item Developed scenario projection models for new products to evaluate capital requirements and embedded value sensitivity under stress assumptions, supporting risk measurement and stakeholder decision-making.
\end{itemize}

\cvsection{Projects}
\textbf{Automated Earnings Call Sentiment and Return Prediction}
\begin{itemize}
\item Engineered a robust text-to-signal pipeline by ingesting transcripts via SEC EDGAR API and parsing with spaCy, reducing production processing time from hours to under 10 minutes per quarter.
\item Applied statistical learning by fine-tuning FinBERT on 5,000+ transcripts to classify tone with 84\% accuracy, then validated signal quality using abnormal-return attribution and cross-sectional testing.
\item Built a stacking ensemble combining NLP features and earnings surprises, improving predictive R\textasciicircum2 by 22\% while maintaining an auditable feature engineering workflow for research-to-production handoff.
\end{itemize}
\vspace{1ex}

\textbf{Credit Risk Modeling for Corporate Bond Portfolios}
\begin{itemize}
\item Calibrated reduced-form hazard rate and Merton structural models to CDS spreads for 300+ issuers, improving consistency of market data inputs used in credit risk measurement workflows.
\item Implemented a correlated-default Monte Carlo engine to estimate CVA and portfolio VaR under historical and stressed correlation assumptions, supporting simulation-based risk analytics and scenario reporting.
\item Performed sensitivity analysis on recovery and default correlation parameters, quantifying impacts on synthetic CDO tranche valuation and highlighting regime-dependent volatility distortions in portfolio loss distributions.
\end{itemize}

\cvsection{Competitions}
\textbf{Morgan Brandley Quant Finance Challenge 2024 (Top 10 Finalist)}
\begin{itemize}
\item Developed a dynamic delta-vega hedging framework for a structured equity product under stochastic volatility, minimizing P\&L variance across 10,000 Monte Carlo paths through systematic rebalancing.
\item Designed a stress-testing suite spanning 8 macro scenarios, translating rate shocks, credit spread widening, and equity drawdowns into coherent risk narratives for portfolio resilience assessment.
\end{itemize}

\end{document}
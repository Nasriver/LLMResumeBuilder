\documentclass[10pt,letterpaper]{article}
\usepackage[utf8]{inputenc}
\usepackage[margin=0.5in]{geometry}

% Disable global paragraph indentation
\setlength{\parindent}{0pt}

% Custom section formatting
\newcommand{\cvsection}[1]{
    \vspace{1.5ex}
    {\noindent\textbf{\large \uppercase{#1}}}
    \vspace{0.3ex}
    \hrule
    \vspace{0.8ex}
}

% Custom list spacing for perfect indentation and ZERO top spacing
\renewenvironment{itemize}{
    \begin{list}{\textbullet}{
        \setlength{\leftmargin}{0.15in}
        \setlength{\itemsep}{1pt}
        \setlength{\parskip}{0pt}
        \setlength{\parsep}{0pt}
        \setlength{\topsep}{0pt}
        \setlength{\partopsep}{0pt}
    }
}{
    \end{list}
}

% Experience entry: \cventry{Company}{Location}{Role}{Dates}
% Uses tabular* to guarantee dates are flush to the right margin.
\newcommand{\cventry}[4]{%
  \noindent\begin{tabular*}{\textwidth}{@{}l@{\extracolsep{\fill}}r@{}}
    \textbf{#1} & #2 \\[0pt]
    \textit{#3} & #4 \\[0pt]
  \end{tabular*}%
}

\pagestyle{empty}

\begin{document}

\begin{center}
    {\LARGE \textbf{WEI (WILLIAM) CHEN}} \\
    999-999-9999 | wchen0924\_NOT\_REAL@gmail.com | https://www.linkedin.com/in/weiwchen\_NOT\_REAL/
\end{center}

\cvsection{Education}
\textbf{UCBB Madison School of Management} \hfill Bar Bara, CA \\
Master of Financial Engineering \hfill GPA: 3.38/4 \hfill December 2025 \\
\textit{Relevant Courses: Derivatives, Fixed-Income Markets, and Econometrics}

\vspace{0.6ex}
\textbf{The English University of Hong Kong} \hfill Hong Kong \\
B.A., Business Administration, Minor in Statistics \hfill December 2022 \\
\textit{Relevant Courses: Time Series, Non-Parametric Statistics, and Deep Learning}

\vspace{1ex}

\cvsection{Skills \& Professional Certifications}
\textbf{Technical Skills:} Python (NumPy, Pandas, Scikit-learn, Polars and Numba), SQL, and Git \\
\textbf{Certifications:} SOA Exam QFIQF \\
\textbf{Other Skills:} Monte Carlo Simulation, Derivatives Pricing, and Machine Learning

\cvsection{Professional Experience}
\cventry{Proton Capital Management}{United States}{Quantitative Research Intern}{June 2025 -- September 2025}
\begin{itemize}
\item Built a systematic screening framework to identify mispriced fixed income securities using yield spread analysis across 500+ instruments, supporting position-level risk interpretation and relative value decision making.
\item Implemented a Python scenario-based interest rate stress testing engine, quantifying DV01 and convexity impacts under parallel shifts, twists, and butterfly moves to explain risk changes across scenarios.
\item Automated daily risk reporting by aggregating key Greeks across the book, reducing manual reporting time by 60\% and eliminating reconciliation errors through consistent data checks.
\end{itemize}
\vspace{1ex}

\cventry{General Life Insurance}{Hong Kong}{Actuarial Analyst}{January 2023 -- September 2024}
\begin{itemize}
\item Designed an experience study framework using 200K+ policy records to update lapse and mortality assumptions, emphasizing statistical validation and clear documentation for governance under IFRS 17 processes.
\item Automated model validation workflows in Python to replace manual Excel checks, reducing quarterly reporting cycle time by 40\% while improving control, reproducibility, and auditability of results.
\item Built scenario projection models for new products, quantifying capital requirements and sensitivity under stressed assumptions to support risk discussions and stakeholder communication.
\end{itemize}

\cvsection{Projects}
\textbf{Credit Risk Modeling for Corporate Bond Portfolios}
\begin{itemize}
\item Calibrated reduced-form hazard rate and Merton structural models to CDS spreads for 300+ issuers, translating market credit factors into issuer-level default risk inputs for portfolio analysis.
\item Built a credit portfolio simulation engine with correlated default processes to estimate portfolio VaR and CVA under historical and stressed correlation assumptions using Monte Carlo methods.
\item Ran sensitivity analysis on recovery rates and default correlations, quantifying parameter impacts on synthetic CDO tranche pricing to support model monitoring and risk explanation.
\end{itemize}
\vspace{1ex}

\textbf{Fixed Income Relative Value and Curve Trading Strategy}
\begin{itemize}
\item Constructed a PCA-based yield curve factor model extracting level, slope, and curvature components, enabling systematic relative value identification across 2s5s10s and butterfly structures.
\item Backtested carry-and-rolldown strategies on US Treasury and swap markets (2015--2024), achieving Sharpe 0.72 gross with max drawdown of $-6.3$\% over the full sample period.
\item Implemented DV01-neutral sizing with stop-loss triggers and z-score mean-reversion filters, improving risk-adjusted returns by 25\% and reducing whipsaw trades by 35\%.
\end{itemize}

\cvsection{Competitions}
\textbf{Morgan Stanley Quant Finance Challenge 2024 (Top 10 Finalist)}
\begin{itemize}
\item Developed a dynamic delta-vega hedging strategy for a structured equity product under stochastic volatility, minimizing P\&L variance across 10,000 Monte Carlo paths with explicit hedge rebalancing logic.
\item Designed a stress testing framework evaluating portfolio resilience under 8 macro scenarios, including rate shocks, credit spread widening, and equity drawdowns, to communicate scenario-driven risk changes.
\end{itemize}

\end{document}
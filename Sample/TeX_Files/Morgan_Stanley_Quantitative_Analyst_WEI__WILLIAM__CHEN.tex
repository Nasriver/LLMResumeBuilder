\documentclass[10pt,letterpaper]{article}
\usepackage[utf8]{inputenc}
\usepackage[margin=0.5in]{geometry}

% Disable global paragraph indentation
\setlength{\parindent}{0pt}

% Custom section formatting
\newcommand{\cvsection}[1]{
    \vspace{1.5ex}
    {\noindent\textbf{\large \uppercase{#1}}}
    \vspace{0.3ex}
    \hrule
    \vspace{0.8ex}
}

% Custom list spacing for perfect indentation and ZERO top spacing
\renewenvironment{itemize}{
    \begin{list}{\textbullet}{
        \setlength{\leftmargin}{0.15in}
        \setlength{\itemsep}{1pt}
        \setlength{\parskip}{0pt}
        \setlength{\parsep}{0pt}
        \setlength{\topsep}{0pt}
        \setlength{\partopsep}{0pt}
    }
}{
    \end{list}
}

% Experience entry: \cventry{Company}{Location}{Role}{Dates}
% Uses tabular* to guarantee dates are flush to the right margin.
\newcommand{\cventry}[4]{%
  \noindent\begin{tabular*}{\textwidth}{@{}l@{\extracolsep{\fill}}r@{}}
    \textbf{#1} & #2 \\[0pt]
    \textit{#3} & #4 \\[0pt]
  \end{tabular*}%
}

\pagestyle{empty}

\begin{document}

\begin{center}
    {\LARGE \textbf{WEI (WILLIAM) CHEN}} \\
    626-491-7382 | wchen0924@gmail.com | https://www.linkedin.com/in/weiwchen/
\end{center}

\cvsection{Education}
\textbf{UCLA Anderson School of Management} \hfill Los Angeles, CA \\
Master of Financial Engineering \hfill GPA: 3.38/4 \hfill December 2025 \\
\textit{Relevant Courses: Fixed-Income Markets, Derivatives, and Econometrics}

\vspace{0.6ex}
\textbf{The Chinese University of Hong Kong} \hfill Hong Kong \\
B.A., Insurance, Financial and Actuarial Analysis, Minor in Statistics \hfill December 2022 \\
\textit{Relevant Courses: Time Series and Non-Parametric Statistics}

\vspace{1ex}

\cvsection{Skills \& Professional Certifications}
\textbf{Technical Skills:} Python (NumPy, Pandas, Scikit-learn, Polars and Numba), SQL, and Git \\
\textbf{Certifications:} SOA Exam QFIQF \\
\textbf{Other Skills:} Monte Carlo Simulation, Derivatives Pricing, and Machine Learning

\cvsection{Professional Experience}
\cventry{Vertex Capital Management}{United States}{Quantitative Research Intern}{June 2025 -- September 2025}
\begin{itemize}
\item Built a high-frequency options data cleaning pipeline to produce audit-ready, consistent inputs for risk and performance analysis across large backtesting parameter grids.
\item Implemented a high-performance Python backtesting framework to evaluate option strategies across varying market regimes, enabling faster iteration on model calibration and systematic strategy selection.
\item Monitored and decomposed daily PnL by strategy and risk drivers, explaining changes driven by position updates and market moves to guide refinements in trading algorithms.
\end{itemize}
\vspace{1ex}

\cventry{Pacific Life Insurance}{Hong Kong}{Actuarial Analyst}{January 2023 -- September 2024}
\begin{itemize}
\item Developed automated VBA pricing tools to evaluate profitability under multiple scenarios, improving evaluation efficiency by 50\% and strengthening governance around assumption changes and model outputs.
\item Performed sensitivity and scenario analysis to quantify impacts of new rider features, communicating results to stakeholders and supporting launch decisions that improved ROE by 10\%.
\item Analyzed 100K+ historical claim records to recalibrate pricing assumptions, strengthening statistical fit, reducing bias, and improving profitability of group health insurance valuations.
\end{itemize}

\cvsection{Projects}
\textbf{Convertible Bond Valuation (UCLA AFP Project with KPMG)}
\begin{itemize}
\item Built convertible bond valuation models (Goldman Sachs and Tsiveriotis--Fernandes) capturing equity dynamics, credit risk, and embedded optionality, aligning pricing outputs with credit-sensitive market risk drivers.
\item Enhanced the framework with explicit recovery rate and credit spread inputs, improving interpretability of valuation impacts from credit factor changes and supporting model methodology documentation.
\item Ran sensitivity analysis on key inputs and confirmed credit spread as the dominant driver, mirroring risk decomposition workflows used for explaining metric changes to stakeholders.
\end{itemize}
\vspace{1ex}

\textbf{Numerical Methods for European, American and Securitized Products}
\begin{itemize}
\item Implemented lattice, finite-difference, and least-squares Monte Carlo pricing for European and American derivatives, validating convergence behavior and estimating Greeks for model testing and performance monitoring.
\item Applied Monte Carlo variance reduction using antithetic and control variates, comparing accuracy and runtime tradeoffs to support robust methodology choices under real-world computational constraints.
\item Valued barrier and default options plus mortgage-backed securities under CIR and G2++ models, strengthening intuition for risk factors across rates, credit-like payoffs, and stress scenarios.
\end{itemize}

\cvsection{Competitions}
\textbf{Annual IAQF Academic Affiliate Membership Student Competition}
\begin{itemize}
\item Developed a 30-day Single-Stock VIX (SSVIX) measure using additional maturities, enabling more reliable risk-neutral volatility estimation when near- and next-term options are illiquid.
\item Improved index implied correlation analysis by integrating SSVIX for top 50 S\&P 500 constituents, producing more theoretically grounded diagnostics for correlation risk and crash risk monitoring.
\item Modeled volatility spillovers with a one-factor Student-t EGARCH and Constant Conditional Correlation framework, simulating market shock impacts and supporting stress-testing style analyses.
\end{itemize}

\end{document}